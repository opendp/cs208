%=================================================================
\documentclass[11pt]{article}

\def\draft{1}

%Packages
\usepackage{amsmath,amssymb,amsthm,enumitem, graphicx,verbatim,hyperref,verbatim,xcolor,rotating,setspace}
\usepackage{algorithm,algorithmic}
\usepackage[top=1in, right=1in, left=1in, bottom=1.5in]{geometry}
\definecolor{spot}{rgb}{0.6,0,0}
% \usepackage[pdftex, bookmarksopen=true, bookmarksnumbered=true,
%   pdfstartview=FitH, breaklinks=true, urlbordercolor={0 1 0},
%   citebordercolor={0 0 1}, colorlinks=true,
%             citecolor=spot,
%             linkcolor=spot,
%             urlcolor=spot,
%             pdfauthor={James Honaker, Salil Vadhan, Jayshree Sarathy},
%             pdftitle={Title}]{hyperref}

% Comments
\ifnum\draft=1
\newcommand{\Snote}[1]{\textcolor{red}{[SV: #1]}}
\newcommand{\JHnote}[1]{\textcolor{violet}{[JH: #1]}}
\newcommand{\ZRnote}[1]{\textcolor{olive}{[ZR: #1]}}
\newcommand{\PNnote}[1]{\textcolor{teal}{[PN: #1]}}
\newcommand{\DAnote}[1]{\textcolor{blue}{[DA: #1]}}
\newcommand{\MSnote}[1]{\textcolor{magenta}{[MS: #1]}}
\newcommand{\CWnote}[1]{\textcolor{brown}{[CW: #1]}}
\newcommand{\YVnote}[1]{\textcolor{orange}{[YV: #1]}}
\newcommand{\CAnote}[1]{\textcolor{blue}{[CA: #1]}}
\newcommand{\JTnote}[1]{\textcolor{purple}{[JT: #1]}}
\newcommand{\SKnote}[1]{\textcolor{cyan}{[SK: #1]}}
\else
\newcommand{\Snote}[1]{}
\newcommand{\JHnote}[1]{}
\newcommand{\ZRnote}[1]{}
\newcommand{\JSnote}[1]{}
\newcommand{\DAnote}[1]{}
\newcommand{\MSnote}[1]{}
\newcommand{\CWnote}[1]{}
\newcommand{\YVnote}[1]{}
\newcommand{\SKnote}[1]{}
\newcommand{\JTnote}[1]{}
\fi 

% Common elements 
\newcommand{\instructions}{\noindent \textbf{Instructions:} Submit a PDF file that contains both your written responses as well as your code to the assignment on Gradescope. Read the section "Collaboration \& AI Policy" in the syllabus for our guidelines regarding the use of LLMs and other AI assistance on the assignments. }

% Theorems
% \theoremstyle{plain}
% \newtheorem{thm}{Theorem}[section]
% \newtheorem{theorem}[thm]{Theorem}
% \newtheorem{lemma}[thm]{Lemma}
% \newtheorem{claim}[thm]{Claim}
% \newtheorem{example}[thm]{Example}
% \newtheorem{exercise}[thm]{Exercise}
% \newtheorem{solution}[thm]{Solution}

\theoremstyle{plain}
\newtheorem{theorem}{Theorem}[section]
\newtheorem{lemma}[theorem]{Lemma}
\newtheorem{corollary}[theorem]{Corollary}
\newtheorem{claim}[theorem]{Claim}
\newtheorem{proposition}[theorem]{Proposition}

\theoremstyle{definition}
\newtheorem{fact}[theorem]{Fact}
\newtheorem{conjecture}[theorem]{Conjecture}
\newtheorem{definition}[theorem]{Definition}
\newtheorem{example}[theorem]{Example}
\newtheorem{remark}[theorem]{Remark}
\newtheorem{question}[theorem]{Question}
% \newtheorem{algorithm}[theorem]{Algorithm}
\newtheorem{openprob}[theorem]{Open Problem}
\newtheorem{exercise}[theorem]{Exercise}
\newtheorem{observation}[theorem]{Observation}

\newtheoremstyle{solution}%
  {\topsep}{\topsep}{\normalfont}{}%
  {\itshape}{.}{5pt}{}
\theoremstyle{solution}
\newtheorem*{solution}{Solution}

% Math macros
\newcommand{\Var}{\mathrm{Var}}
\newcommand{\Exp}{\mathrm{E}}
\newcommand{\R}{\mathbb{R}}
\newcommand{\N}{\mathbb{N}}
\newcommand{\Normal}{\mathcal{N}}
\newcommand{\Bin}{\mathrm{Bin}}
\newcommand{\Bern}{\mathrm{Bern}}
\newcommand{\Lap}{\mathrm{Lap}}
\newcommand{\Shuffle}{\mathrm{Shuffle}}
\newcommand{\naturals}{\mathbb{N}}

\newcommand{\calA}{\mathcal{A}}
\newcommand{\calN}{\mathcal{N}}
\newcommand{\calR}{\mathcal{R}}
\newcommand{\calT}{\mathcal{T}}
\newcommand{\calX}{\mathcal{X}}
\newcommand{\calY}{\mathcal{Y}}

\newcommand{\reals}{\mathbb{R}}
\newcommand{\eps}{\varepsilon}
\newcommand{\Range}{\mathrm{Range}}
\newcommand{\Supp}{\mathrm{Supp}}
\def\norm#1{\mathopen\| #1 \mathclose\|}% use instead of $\|x\|$
\newcommand{\brackets}[1]{\langle #1\rangle}

% PUMS macros
\newcommand{\data}{\texttt{data}}
\newcommand{\pub}{\texttt{pub}}
\newcommand{\pubA}{\texttt{alice}}
\newcommand{\us}{\texttt{uscitizen}}
\newcommand{\sex}{\texttt{sex}}
\newcommand{\age}{\texttt{age}}
\newcommand{\educ}{\texttt{educ}}
\newcommand{\married}{\texttt{married}}
\newcommand{\divorced}{\texttt{divorced}}
\newcommand{\latino}{\texttt{latino}}
\newcommand{\black}{\texttt{black}}
\newcommand{\asian}{\texttt{asian}}
\newcommand{\children}{\texttt{children}}
\newcommand{\employed}{\texttt{employed}}
\newcommand{\militaryservice}{\texttt{militaryservice}}
\newcommand{\disability}{\texttt{disability}}
\newcommand{\englishability}{\texttt{englishability}}

\newcommand{\zo}{\{0,1\}}

% Macros from previous section notes
\newcommand{\sol}{\noindent \textit{Solution.}~}
\newcommand{\ezdp}{$(\epsilon,0)$-differentially private}
\newcommand{\ezdpy}{$(\epsilon,0)$-differential privacy}
\newcommand{\eddpy}{$(\epsilon, \delta)$-differential privacy}
\newcommand{\eddp}{$(\epsilon, \delta)$-differentially private}

\newcommand{\cH}{\mathcal{H}}
\newcommand{\cG}{\mathcal{G}}
\newcommand{\GS}{\mathrm{GS}}
\newcommand{\LS}{\mathrm{LS}}
\newcommand{\RS}{\mathrm{RS}}

\newcommand{\sort}{\mathrm{sort}}

\newcommand{\cR}{\mathcal{R}}
\newcommand{\cX}{\mathcal{X}}
\newcommand{\cY}{\mathcal{Y}}
\newcommand{\cZ}{\mathcal{Z}}
\newcommand{\cD}{\mathcal{D}}
\newcommand{\cI}{\mathcal{I}}
\newcommand{\cE}{\mathcal{E}}
\newcommand{\cS}{\mathcal{S}}

\newcommand{\eqdef}{\stackrel{\text{def}}{\scalebox{2}[1]{=}}}

\newcommand{\Ham}{\textit{Ham}}
\newcommand{\Sym}{\textit{Sym}}
\newcommand{\Abs}{\textit{Abs}}
\newcommand{\IM}{\textit{IM}}
\newcommand{\OM}{\textit{OM}}
\newcommand{\User}{\textit{User}}

\newcommand{\userid}{\texttt{userid}}
\newcommand{\event}{\texttt{event}}


\title{\vspace{-1.5cm} HW 4: Differential Privacy Foundations}
\author{CS 208 Applied Privacy for Data Science, Spring 2025}
\date{\textbf{Version 1.2: Due Fri, Feb. 28, 5:00pm.}}


%=================================================================

\begin{document}
\maketitle

\instructions

\begin{enumerate}[leftmargin=*]

\item \textbf{User-level vs. Event-level Privacy.} 
Recall that differential privacy is defined with respect to a dataset space $\cX$ and an adjacency relation $\sim$
on $\cX$ that determines our {\em privacy unit}: DP protects information that can differ between adjacent datasets.

A more general formulation uses dataset {\em metrics} $d : \cX\times\cX \rightarrow \R\cup \{\infty\}$ as covered in HoDP Chapter 3.  Given a dataset metric, we can obtain an adjacency relation by defining $x\sim_d x'$ if $d(x,x')\leq 1$.
We have seen two examples for the dataset space $\cX = \cR^* \eqdef \bigcup_{n=0}^\infty \cR^n$, where $\cR$ is the space from which individual records (or rows) come:\footnote{In class, we often used $\cX$ for the row space, and $\cX^n$ for the dataset space.} 
\begin{itemize}
\item {\em Hamming distance}: if we have two datasets $x,x'\in \cR^*$ such that $|x|=|x'|=n$ (where $|x|$ is the number of records in $x$), then $d_\Ham(x,x') \eqdef |\{i : x_i\neq x'_i\}|$.  If $|x|\neq |x'|$, then we define $d_\Ham(x,x')=\infty$. The induced adjacency relation $\sim_\Ham$ is the ``change one row'' notion of adjacency, which we used to formalize {\em bounded DP} where the number $n$ of records is treated as publicly known.
\item {\em Symmetric distance}: if we have two datasets $x,x'\in \cR^*$, then $d_\Sym(x,x')$ is the size of the symmetric difference between $x$ and $x'$ as multisets, i.e. the number of records that need to be added or removed from $x$ for it to have the same number of occurrences of each element as $x'$.  $\sim_\Sym$ is the ``add or remove a row'' notion of adjacency, which we used to formalize {\em unbounded DP}, where the number $n$ of records needs also to be protected with DP.
\end{itemize}
Another difference between $d_\Ham$ and $d_\Sym$ is that the former depends on the order of elements in $x$ and $x'$ and the latter does not. It is possible to define other metrics to bridge this distinction, but we do not do so here.

Recall that a transformation $T : \cX \rightarrow \cY$ between two metric spaces (with input metric $d_\IM$ and output metric $d_\OM$) is called {\em $c$-stable} if for all $x,x'\in \cX$, we have $d_\OM(T(x),T(x'))\leq c\cdot d_\IM(x,x')$.  
Stable transformations are useful because they compose well with differentially private mechanisms: If $T$ as above is $c$-stable with respect to $d_\IM$ and $d_\OM$ and $M : \cY\rightarrow \cZ$ is $\eps$-DP with respect to $\sim_\OM$, then their chaining $(M\circ T)(x) \eqdef M(T(x))$ is $c\eps$-DP.  We can understand the privacy analysis of the Laplace mechanism using this fact, by taking $\cY=\R$, $d_\OM=d_\Abs$, and $M(y)=y+\Lap(1/\eps)$.  Then $T$ being $c$-stable is the same as saying its global sensitivity is bounded as $\Delta T\leq c$, and $M$ being $\eps$-DP with respect to $\sim_\Abs$, amounts to saying that the pdfs of $y+\Lap(1/\eps)$ and $y'+\Lap(1/\eps)$ are within an $e^\eps$ factor if $|y-y'|\leq 1$.

We can use this formalism to distinguish ``user-level'' and ``event-level'' privacy and reason about ``contribution bounding'' as often comes up in DP deployments (including the Wikimedia release you read about).
In the rest of this problem, take $\cX=\cR^*$ for a row space of the form $\cR=\cI\times \cE$, where $\cI$ is a space of userids and $\cE$ is space of events associated with a user (e.g. a user edited a particular Wikipedia page at a particular time). So each record is of the form $(\userid,\event)$ and the same $\userid$ can appear arbitrarily many times in the dataset. 

For two datasets $x,x'\in \cX$, we can define $d_\User(x,x')$ to be the number of additions or removals of {\em users} from $x$ for it to have the same number of occurrences of each element (both user and event) as $x'$. Here removing a user means removing all records with a given $\userid$, and adding a user means adding all records with a given $\userid$.  {\em User-level privacy} means DP with respect to $\sim_\User$. In contrast, {\em event-level privacy} simply means DP with respect to the usual $\sim_\Sym$ or $\sim_\Ham$ in the case of bounded DP.

\begin{enumerate}
\item \label{part:sensitivity} Consider the a query 
$q : \cX\rightarrow \R$ defined by 
$q(x) = |x|$.  What is the global sensitivity $\Delta q$ with respect to $\sim_\User$ and with respect to $\sim_\Sym$?

\item \label{part:stability} Assume that there exists a total ordering on events in $\mathcal{E}$ (e.g., a timestamp for each event). Consider the transformation $T : \cX\rightarrow \cX$ where $T(x)$ keeps only the smallest (under the total ordering on $\mathcal{E}$) $\leq B$ records associated with each \userid\ and just drops the remaining ones.  For each pair of metrics $d_\IM,d_\OM \in \{d_\Sym,d_\Ham,d_\User\}$, calculate the smallest $c$ such that $T$ is a $c$-stable transformation from $d_\IM$ to $d_\OM$.  

\item Combining Part~\ref{part:sensitivity} and Part~\ref{part:stability} with $\IM=\User$ and an appropriate choice of $\OM$, deduce a bound on global sensitivity of $(q\circ T) : \cX\rightarrow \R$ with respect to $\sim_\User$.
\end{enumerate}

\item \textbf{Regression: } 
Consider a dataset where each of its $n$ rows is a pair of real numbers $(x_i,y_i)$.  Suppose we wish to find a best-fit linear relationship $y_i \approx \beta x_i$ between the $y$'s and the $x$'s.  Non-privately, a standard way to estimate $\beta$ is via the OLS regression formula 
$$\hat{\beta} = \hat{\beta}(x,y) =  \frac{S_{xy}}{S_{xx}}
= \frac{\sum_i x_iy_i}{\sum_i x_i^2}.$$
This is called {\em ordinary least-squares (OLS)} regression, since $\hat{\beta}$ is the minimizer of the mean-squared residuals 
\begin{equation} \label{eqn:residuals}
    \frac{1}{n} \sum_i (y_i -\hat{\beta} x_i)^2.
\end{equation}

\begin{enumerate}
    \item Show that, even if we restrict the data domain so that the $x_i,y_i\in [-b,b]$ for a finite $b\geq 0$, the function $\hat{\beta}(x,y)$ has infinite global sensitivity with respect to $\sim_\Ham$,
    and hence we cannot get a useful DP estimate of it via a direct application of the Laplace or Gaussian mechanisms.
    \item In contrast, show that $S_{xy}$ and $S_{xx}$ have global sensitivity with respect to $\sim_\Ham$ on $([-b,b]\times [-b,b])^n$ that is bounded solely as a function of $b$, and hence each of these can be approximated in a DP manner using the Laplace mechanism. \label{part:suffstats}
    \item Using Part~\ref{part:suffstats} together with a stable clipping transformation, chaining,  
    basic composition, and post-processing,  devise and implement an $\eps$-DP algorithm (wrt $\sim_\Ham$)  for approximating $\hat{\beta}$ on an arbitrary dataset with $x_i,y_i\in \R$.  
    In addition to the dataset $((x_1,y_1),\ldots,(x_n,y_n))$, your implementation should take as input parameters a clipping bound $b$ and the privacy-loss parameter $\eps$.
    
    \item \label{part:MonteCarlo}
    Evaluate the performance of your
    algorithm using a Monte Carlo simulation with
    synthetic data.
    Set $\eps=.1$, $b=1$, generate the $x_i$'s uniformly at random from $[-1/2,1/2]$,
    and generate the $y_i$'s 
    according to a linear model with slope 1 and Gaussian noise:
    $$y_i = x_i + \mathcal{N}(0,.02)$$
    For each $n=100,200,300,\ldots,5000$, run many Monte Carlo trials to estimate and plot the bias and standard deviation of both the OLS estimate $\hat{\beta}$ and the DP estimate $\tilde{\beta}$.  Your plot should have $n$ on the $x$-axis, and bias and standard deviation on the $y$-axis on a scale from $-1.0$ to $1.0$, with the endpoints also representing values of magnitude larger than 1.  Try to run enough trials to obtain smooth curves.
    
    (If $\hat{\theta}=\hat{\theta}(z)$ is an estimator of a  population parameter $\theta$ based on a dataset $z$, then the {\em bias} of $\hat{\theta}$ is $\Exp[\hat{\theta}-\theta]$, where the expectation is taken over both the dataset $z$ and any randomization used by estimator $\hat{\theta}$; note that the bias can be positive or negative---do keep track of the sign.  The ``bias-variance tradeoff'' says that the MSE of an estimator is the sum of its squared bias and its variance; in previous homeworks, we evaluated the (R)MSE of DP estimators, now we are doing a finer analysis by separating the MSE into the bias and variance.)
    
    \item Try to give an intuitive explanation of the source of the bias you see in 
    Part~\ref{part:MonteCarlo} and on what kinds of dataset distributions this might be largest.  How might bias in particular (not just MSE) have an impact on downstream applications?
\end{enumerate}
\item \textbf{Final Project Ideas} 
The final projects are an important focus of this course, and we want you to start thinking about yours as soon as possible.  Please read the
``Final Project Guidelines'' (to be posted shortly on the course webpage: \url{https://github.com/opendp/cs208/blob/main/spring2025/final%20projects/Final%20Project%20Guidelines%202025.pdf}) 
and submit about a paragraph as described in the ``Topic Ideas'' bullet.

\end{enumerate}

\subsection*{Collaborators}
Please list all collaborators for this problem set. ChatGPT and other AI tools should be treated similarly to collaboration with your peers in the 
class.  You may use these tools to help you understand the material and as part of your 
brainstorming process, but you should not be asking the tools to solve the homework problems 
for you. If you do use such tools, you must cite them and  list the 
prompts you entered and responses obtained below.

\end{document}
